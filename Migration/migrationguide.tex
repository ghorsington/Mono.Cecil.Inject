% !TeX spellcheck = en_GB
\documentclass[a4paper,11pt]{article}
\usepackage[margin=1in]{geometry}
\usepackage{minted}
\usepackage{framed}
\usepackage[usenames, dvipsnames]{color}
\usepackage[colorlinks=true,urlcolor=NavyBlue]{hyperref}
\usepackage{tcolorbox}

\definecolor{vsblue}{HTML}{007ACC}
\definecolor{vsgrey}{HTML}{293955}
\definecolor{vsbg}{HTML}{F9F9FC}

\tcbuselibrary{breakable, minted}

\usemintedstyle{vsnew}
\newminted[cs]{csharp}{autogobble, breaklines}
\setlength{\parindent}{0pt}
\setlength{\parskip}{1em}

\newtcolorbox{mdef}{ %
breakable,
title=\textbf{METHOD DEFINITION},
title filled=false,
colframe=vsgrey,
colback=vsbg,
halign title=flush right,
arc=0pt,
subtitle style={colback=vsblue, boxrule=0pt,fontupper=\sc}
}

\newcommand{\CecilInject}{\textsc{C}{\scriptsize \sc ecil}.\textsc{I}{\scriptsize \sc nject}}

\begin{document}
\title{\textsc{C}{\large \sc ecil}.\textsc{I}{\large \sc nject} \\ \vspace{-10pt} {\Large Porting code from ReiPatcherPlus}}

\author{Geoffrey ''denikson`` Horsington \\
{\normalsize \tt \href{https://github.com/denikson}{Github}} \\
{\normalsize \tt \href{http://www.hongfire.com/forum/member.php/1347954-denikson}{HongFire}}
}
\maketitle
\section{Abstract}
ReiPatcherPlus is an extension library to ReiPatcher that provides simple injection methods for Mono.Cecil. The initial versions (1.x) provided very basic means of hooking and redirecting methods. The updated versions (2.x) also allowed to explicitly search for methods and hooks based on method name, parameters and additional flags. 
However, ReiPatcherPlus is lacking in several ways:
\begin{itemize}
\item Lack of additional methods, like method search by \texttt{System.Type} parameters and more advanced access changing methods
\item Poor method hierarchy and old methods left for the sake of compatibility
\item Dependency on ReiPatcher even though almost no methods use it
\end{itemize}
To address these issues, nearly every method in ReiPatcherPlus was revised, rewritten and fixed where needed. Moreover, the dependency on ReiPatcher was removed and all functionality moved to separate classes to extend Mono.Cecil. Thus, being a massive update and an extension to Mono.Cecil instead of ReiPatcher, the library was renamed to Mono.Cecil.Inject (hereafter \CecilInject{}) to better represent the purpose the library.

This document is a quick reference guide for those who come from ReiPatcherPlus and wish to port their code to \CecilInject{}. Only the most crucial changes are outlined and only the basic set of new methods are shown. Old methods from ReiPatcherPlus and new ones from \CecilInject{} are shown side-by-side for easier comparison. The exact definitions of \emph{some} new methods are picked from the official documentation of \CecilInject{} and are displayed in framed boxes. For the complete list of all methods in \CecilInject{}, refer to the official documentation.
\newpage
\section{Importing the library}
The easiest way to begin migration is to remove the reference to ReiPatcherPlus, add the reference to \CecilInject{} and replace the old namespace import with the new one:
\begin{cs}
using Mono.Cecil.Inject;
\end{cs}
Attempting to compile the project will cause numerous errors; use them to find the code that requires migration.

\section{Loading assemblies}
ReiPatcherPlus provided the following method to load assemblies into Mono.Cecil:
\begin{cs}
AssemblyDefinition LoadAssembly(this PatchBase patch, string name);
\end{cs}
\vspace{-1em}
As the method signature suggests, the method is an extension to \texttt{PatchBase} class found in ReiPatcher. Because of removed dependency on ReiPatcher, the functionality of the method was changed a bit. The new definition found in \CecilInject{} is as follows:
\begin{mdef}
In \texttt{AssemblyLoader}:
\begin{cs}
public static AssemblyDefinition LoadAssembly(string path);
\end{cs}
\tcbsubtitle{Description} 
Loads the assembly specified by path into Mono.Cecil.
\tcbsubtitle{Parameters}
\begin{itemize}
\item[$\triangleright$] \texttt{path} -- Path to the assembly to load. Can be absolute or relative. In the latter case the path must be relative to the executing assembly which invokes this method.
\end{itemize}
\tcbsubtitle{Returns} 
An instance of \texttt{AssemblyDefinition} of the loaded assembly.
\tcbsubtitle{Examples}
\begin{cs}
	AssemblyDefinition ad1 = AssemblyLoader.LoadAssembly("SBPR.HelloWorld.dll");
	
	// NOTE: This is equivalent of ReiPatcherPlus' method call:
	// this.LoadAssembly("CM3D2.HelloWorld.Hook.dll");
	AssemblyDefinition ad2= AssemblyLoader.LoadAssembly(Path.Combine(AssembliesDir, "CM3D2.HelloWorld.Hook.dll"));
\end{cs}
\end{mdef}
\newpage
Note especially that
\begin{itemize}
\item This method is not an extension any more and therefore must be referred to with the class name (\texttt{AssemblyLoader})
\item The root of the relative path is now the directory of the executing assembly (in case of ReiPatcher it is the directory ReiPatcher.exe lies in) instead of the game assembly directory. Remember to use \texttt{Path.Combine} with property \texttt{AssembliesDir} provided by \texttt{PatchBase} in order to achieve the same functionality as before.
\end{itemize}

\section{Checking patch flags}
ReiPatcherPlus provided the following method to check for patch attributes:
\begin{cs}
bool HasAttribute(this PatchBase patch, 
                  AssemblyDefinition assembly, 
                  string attribute);
\end{cs}
In \CecilInject{} the method \textbf{has been removed} because it was only usable for ReiPatcher. As of this moment no features to add similar functionality is planned. Instead, you can use the following method to replicate the behaviour of \texttt{HasAttribute} found in ReiPatcherPlus:
\begin{cs}
public static bool HasAttribute(PatchBase patch, 
                                AssemblyDefinition assembly, 
                                string attribute)
{
	return patch.GetPatchedAttributes(assembly)
                    .Any(a => a.Info == attribute);
}
\end{cs}

\section{\texttt{TypeDefinition} extensions}
Many of the extension methods for \texttt{TypeDefinition} of Mono.Cecil were ported from ReiPatcherPlus to \CecilInject{}.
Namely, the following familiar methods are found in \CecilInject{}:
\begin{cs}
MethodDefinition GetMethod(this TypeDefinition self, string methodName);
MethodDefinition GetMethod(this TypeDefinition self,
                           string methodName,
                           params TypeReference[] paramTypes);
FieldDefinition GetField(this TypeDefinition self, string memberName);
\end{cs}
In addition, \texttt{ChangeAccess} has been moved, fixed and has some new parameters. Because of this, we shall take a closer look at this function.
\begin{mdef}
Extension to \texttt{TypeDefinition} (in \texttt{TypeDefinitionExtensions}):
\begin{cs}
public static void ChangeAccess(this TypeDefinition type,
                                string member,
                                bool makePublic = true,
                                bool makeVirtual = true,
                                bool makeAssignable = true,
                                bool recursive = false);
\end{cs}
\tcbsubtitle{Description}
A method to quickly make any class member public and writeable.
\tcbsubtitle{Parameters}
\begin{itemize}
\item[$\triangleright$] \texttt{type} -- type the members of which to modify. Will be inferred by the compiler if used as an extension method.
\item[$\triangleright$] \texttt{member} -- the name of the member to modify. Can be either set to a literal name or a regular expression.
\item[$\triangleright$] \texttt{makePublic} -- make the member public if it isn't already.
\\ \textit{Default:} \texttt{true}
\item[$\triangleright$] \texttt{makeVirtual} -- make the member virtual, if it is a method.
\\ \textit{Default:} \texttt{true}
\item[$\triangleright$] \texttt{makeAssignable} -- make the member assignable, if it is marked as \texttt{readonly}.
\\ \textit{Default:} \texttt{true}
\item[$\triangleright$] \texttt{recursive} -- If set to \texttt{true} will recursively apply this method to every nested class in the type.
\\ \textit{Default:} \texttt{false}
\end{itemize}
\tcbsubtitle{Remarks}
Note that you can now select the members with regular expressions.
\newline

The method does not make the base type (specified in \texttt{type} parameter) public, since it can be done easily by using Mono.Cecil itself.
\tcbsubtitle{Examples}
\begin{cs}
TypeDefinition myType = myAssembly.MainModule.GetType("MyType");

// This is the intended way to use this method
myType.ChangeAccess("someMember1");

// This also works, but is not advised
TypeDefinitionExtensions.ChangeAccess(myType, "someMember1");

// You can use RegEx to match multiple members
// You can also set one or many optional parameter(s) by either referencing them by name in any order or by assigning them values in the order they appear in the signature
myType.ChangeAccess("someMember.*", makeAssignable: false);
\end{cs}
\end{mdef}

To make \CecilInject{} more compatible with System.Reflection, new methods were added. Here are some the definitions

\begin{mdef}
Extension to \texttt{TypeDefinition} (in \texttt{TypeDefinitionExtensions}):
\begin{cs}
public static MethodDefinition GetMethod(this TypeDefinition self, 
                                         string methodName, 
                                         params Type[] types);
\end{cs}
\tcbsubtitle{Description}
Searches for the method with the given name and parameter types.

\tcbsubtitle{Parameters}
\begin{itemize}
\item[$\triangleright$] \texttt{self} -- type that contains the method to search. Will be inferred by the compiler if used as an extension method.
\item[$\triangleright$] \texttt{methodName} -- name of the method to find.
\item[$\triangleright$] \texttt{types} -- type of parameters in the method to search. Must be specified in the order they are defined in the method signature.
\end{itemize}

\tcbsubtitle{Returns}
An instance of \texttt{MethodDefinition} for the found method. Returns \texttt{null}, if no fitting method found.

\tcbsubtitle{Remarks}
Use \texttt{System.Type} to construct types. 
\newline

If generic types are needed, use \texttt{ParamHelper.CreateDummyType} to easily create parameters with custom names.

\tcbsubtitle{Examples}
\begin{cs}
TypeDefinition myType = myAssembly.MainModule.GetType("MyType");

// Searches for method signature: MyMethod1(int, ref bool) (or MyMethod1(int, out bool))
MethodDefiniton md1 = myType.GetMethod("MyMethod1", 
                                       typeof(int), 
                                       typeof(bool).MakeByReferenceType());

// Searches for method signature: MyMethod2<T>(List<T>);
MethodDefinition md2 = 
  myType.GetMethod("MyMethod2", 
     typeof(List<>).MakeGenericType(ParamHelper.CreateDummyType("T")));
\end{cs}
\end{mdef}
\newpage
There is also a version to retrieve all the overloads of a method

\begin{mdef}
Extension to \texttt{TypeDefinition} (in \texttt{TypeDefinitionExtensions}):
\begin{cs}
public static MethodDefinition[] GetMethods(this TypeDefinition self, 
                                            string methodName);
\end{cs}

\tcbsubtitle{Description}
Retrieves all the overloads of the given method name.

\tcbsubtitle{Parameters}
\begin{itemize}
\item[$\triangleright$] \texttt{self} -- type that contains the methods to search. Will be inferred by the compiler if used as an extension method.
\item[$\triangleright$] \texttt{methodName} -- name of the method to search for.
\end{itemize}

\tcbsubtitle{Returns}
An array of \texttt{MethodDefinition} containing all the methods the name of which is \texttt{methodName}.

\tcbsubtitle{Examples}
\begin{cs}
TypeDefinition myType = myAssembly.MainModule.GetType("MyType");

// Gets all overloads of MyMethod
MethodDefinition[] mds = myType.GetMethods("MyMethod");
\end{cs}
\end{mdef}

\section{\texttt{Parameter} is now \texttt{ParamHelper}}
To create generic types for \texttt{GetMethod}, ReiPatcherPlus had a helper class \texttt{Parameter}. In \CecilInject{}, such class is still left, but some of the methods have been removed so as to force the developers to prefer the newer version of \texttt{GetMethod} described in the previous section.

More exactly, the following methods were ported from ReiPatcherPlus:
\begin{cs}
public static TypeReference CreateGeneric(string name);
public static TypeReference FromType<T>();
public static TypeReference FromType(Type type);
\end{cs}

Note that \texttt{FromType} takes only one argument now. In order to construct reference and/or generic types, use the methods provided by \texttt{System.Type}. However, to aid the creation of type with generic types (like \texttt{List<T>}), one can use the following method in \texttt{ParamHelper}:
\begin{cs}
public static Type CreateDummyType(string name);
\end{cs}
which allows to create a generic type with a custom name.

\section{\texttt{MethodHook} is now \texttt{InjectionDefinition}}
Just like in ReiPatcherPlus, method injection is the main feature of \CecilInject{}. Therefore, more care was put into reorganising the injection methods. The main differences and updates are found in the following subsections.

\subsection{\texttt{MethodFeatures} is \texttt{InjectFlags}}
\texttt{MethodFeatures} -- the enumeration containing flags that specify how to inject method -- has been renamed into \texttt{InjectFlags} to better represent its meaning. Additionally, the flags have been renamed to better represent their nature. To summarise, below is the table that shows how the names have been changed.
\begin{center}
\begin{tabular}{|l|l|}
\hline
Old flag name in \texttt{MethodFeatures} & New flag name in \texttt{InjectFlags} \\ \hline
\texttt{None} & \texttt{None} \\ \hline
\texttt{PassCustomTag} & \texttt{PassTag} \\ \hline
\texttt{PassTargetType} & \texttt{PassInvokingInstance} \\ \hline
\texttt{PassReturn} & \texttt{ModifyReturn} \\ \hline
\texttt{PassMemberReferences} & \texttt{PassFields} \\ \hline
\texttt{PassLocalReferences} & \texttt{PassLocals} \\ \hline
\texttt{PassMethodParametersByValue} & \texttt{PassParametersVal} \\ \hline
\texttt{PassMethodParametersByReference} & \texttt{PassParametersRef} \\ \hline
\texttt{All\_ByValue} & \texttt{All\_Val} \\ \hline
\texttt{All\_ByReference} & \texttt{All\_Ref} \\ \hline
\end{tabular}
\end{center}
Other than changed names, they produce the identical injection results.

\subsection{Extensions to \texttt{InjectFlags}}
One of the new additions to \CecilInject{} is the ability ease the construction and manipulation of \texttt{InjectFlags}. Namely, the following extension methods were added to \texttt{InjectFlags}:
\begin{cs}
public static bool IsSet(this InjectFlags flags, InjectFlags flag);
public static InjectValues ToValues(this InjectFlags flags);
\end{cs}

Moreover, a structure \texttt{InjectValues} was added that can be used to represent \texttt{InjecFlags} as boolean parameters and vice versa.

We shall omit the exact documentation -- feel free to experiment yourself or consult the official documentation.

\subsection{Obtaining \texttt{InjectionDefinition}}
To get an instance of \texttt{InjectionDefinition}, you can do one of the following
\begin{itemize}
\item Create an instance through the constructor of \texttt{InjectionDefinition} 
\\ (equivalent to ReiPatcherPlus' \texttt{MethodHook.FromMethodDefinition})
\item Use the extension method to \texttt{TypeDefintion} which is \texttt{GetInjectionMethod} (equivalent to ReiPatcherPlus' \texttt{GetHookMethod}).
\item Use \texttt{GetInjector} (extension method to \texttt{MethodDefinition}) to get an instance of \\ \texttt{InjectionDefinition}. Like \texttt{GetInjectionMethod}, the method searches for a fitting hook in the provided \texttt{type}.
\end{itemize}
The exact method signatures are the same as in ReiPatcherPlus and therefore they shall be omitted. It shall be noted, however, that the behaviour of \texttt{GetInjectionMethod} differes a little bit from that of \texttt{GetHookMethod}. Namely
\begin{enumerate}
\item The method is an extension to \texttt{TypeDefinition} now. Therefore, \texttt{hookType} is the first parameter that is inferred by the compiler when you call the method on some \texttt{TypeDefinition}.
\item The method does not attempt to fix obvious issues with the specified \texttt{InjectFlags}. Instead, when any mismatch occurs, the method simply returns \texttt{null}.
\end{enumerate}

\subsection{Injection}
Owing to compatibility, ReiPatcherPlus contained a plethora of different overloads intended for method injection. In \CecilInject{}, all the methods were combined into two.
Since they are the most important methods, here are their full definitions:
\begin{mdef}
In \texttt{InjectionDefinition}
\begin{cs}
public void Inject(int startCode = 0,
                   int token = 0, 
                   InjectDirection direction = InjectDirection.Before);
                   
public void Inject(Instruction startCode, 
                   int token = 0, 
                   InjectDirection direction = InjectDirection.Before);
\end{cs}

\tcbsubtitle{Description}
Applies the injection specified in \texttt{InjectionDefinition}.

\tcbsubtitle{Parameters}
\begin{itemize}
\item[$\triangleright$] \texttt{startCode} -- the instruction in the method from which to start to inject. Can be specified either as an index (integer) or as an \texttt{Instruction}.
\item[$\triangleright$] \texttt{token} -- if the hook method can receive custom tags (specified with \texttt{InjecFlags.PassTag}), this is the tag that will be passed to it.
\item[$\triangleright$] \texttt{direction} -- the direction in which to inject the method call. The injection can be either done before the specified \texttt{startCode} or after it.
\end{itemize}

\phantom{a}
\tcbsubtitle{Remarks}
If \texttt{startCode} is specified as an index, it can be either positive (specifies code from the top of the method) or negative (specifies code from the end of the method).

\tcbsubtitle{Examples}
\begin{cs}
TypeDefinition myHooks = myAssembly.GetType("MyHooks);
TypeDefinition myTarget = targetAssembly.GetType("TargetType");

MethodDefinition target = myTarget.GetMethod("MyTargetMethod");

// Searches for hook with signature MyHookMethod() and injects it in front of the first instruction of target method
myHooks.GetInjectionMethod("MyHookMethod", 
                           target, 
                           InjectFlags.None).Inject();
                           
// ALTERNATIVE: Create InjectionDefinition manually
InjectionDefinition id = 
     new InjectionDefinition(target,
                             myHooks.GetMethod("MyHookMethod"),
                             InjectFlags.None);
id.Inject();                                     
\end{cs}
\end{mdef}

Alternatively, you can use the following extension to \texttt{MethodDefinition} to inject with a single expression.

\begin{mdef}
Extension to \texttt{MethodDefinition} (in \texttt{MethodDefinitionExtensions}):
\begin{cs}
public static void InjectWith(this MethodDefinition method,
                        MethodDefinition injectionMethod,
                        int codeOffset = 0,
                        int tag = 0,
                        InjectFlags flags = InjectFlags.None,
                        InjectDirection dir = InjectDirection.Before,
                        int[] localsID = null,
                        FieldDefinition[] typeFields = null);
\end{cs}

\tcbsubtitle{Descriptipn}
Creates an instance of \texttt{InjectionDefinition} and calls \texttt{Inject}.

\tcbsubtitle{Parameters}
\begin{itemize}
\item[$\triangleright$] \texttt{method} -- the method which to inject. Will be inferred by the compiler if used as an extension method.
\item[$\triangleright$] \texttt{injectionMethod} -- the hook method. The call to this method will be injected into \texttt{method}.
\item[$\triangleright$] \texttt{codeOffset} -- the index of the instruction in \texttt{method} from which to start injecting.
\item[$\triangleright$] \texttt{tag} -- if the hook method can receive custom tags (specified with \texttt{InjecFlags.PassTag}), this is the tag that will be passed to it.
\item[$\triangleright$] \texttt{flags} -- the flags which specify what and how to inject the method. Different flags can be combined with a logical OR (\texttt{|}) or by using \texttt{InjectValues}.
\item[$\triangleright$] \texttt{dir} -- the direction in which to inject the method call. The injection can be either done before the specified \texttt{startCode} or after it.
\item[$\triangleright$] \texttt{localID} -- an array of method local indices to pass to the hook. Used only if \texttt{InjectFlags.PassLocals} is specified.
\item[$\triangleright$] \texttt{typeFields} -- Fields to pass to the hook. Used only if \texttt{InjectFlags.PassFields} is specified.
\end{itemize}

\tcbsubtitle{Remarks}
Acts exactly like \texttt{AttachMethod} in ReiPatcherPlus.

\tcbsubtitle{Examples}
\begin{cs}
TypeDefinition myHooks = myAssembly.GetType("MyHooks);
TypeDefinition myTarget = targetAssembly.GetType("TargetType");

// Injects MyTargetMethod with (for example) MyHook1(ref int myLocal, ref int tField)
myTarget.GetMethod("MyTargetMethod").InjectWith(
                   myHooks.GetMethod("MyHook1"),
                   flags: InjectFlags.PassLocals
                        | InjectFlags.PassFields,
                   localsID: new[] {0},
                   typeFields: new[] {myTarget.GetField("tField")});
\end{cs}
\end{mdef}

\section{Summary}
We discussed the most important changes between ReiPatcherPlus and \CecilInject{} thus making migration between the libraries relatively easy. If you wish to inspect the methods of \CecilInject{} more thoroughly and get a better explanation on how to use the library, consider reading the full documentation of \CecilInject{}.
\end{document}